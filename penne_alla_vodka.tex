\begin{recipe}{Penne alla Vodka}{6 servings - 565 calories}{}

\freeform From Epicurious - Lidia's Italian-American Kitchen.

\Ing{Salt}
\Ing{One 35-ounce can Italian plum tomatoes (preferably San Marzano) with their liquid}
\Ing{1 pound penne}
\Ing{\nicefrac{1}{4} cup extra-virgin olive oil}
\Ing{10 cloves garlic, peeled}
\Ing{Crushed hot red pepper}
\Ing{\nicefrac{1}{4} cup vodka}
\Ing{\nicefrac{1}{2} cup heavy cream}
\Ing{2 tablespoons unsalted butter or olive oil for finishing the sauce, if you like}
\Ing{2 to 3 tablespoons chopped fresh Italian parsley}
\Ing{\nicefrac{3}{4} cup freshly grated Parmigiano-Reggiano, plus more for passing if you like}

Cook the penne. Meanwhile, pour the tomatoes and their liquid into the work bowl of a food processor. Using quick on/off pulses, process the tomatoes just until they are finely chopped. (Longer processing will aerate the tomatoes, turning them pink.)\\

Meanwhile, heat the olive oil in a large skillet over medium heat. Whack the garlic cloves with the side of a knife and add them to the hot oil. Cook, shaking the skillet, until the garlic is lightly browned, about 3 minutes. Lower the work bowl with the tomatoes close to the skillet and carefully — they will splatter — slide the tomatoes into the pan. Bring to a boil, season lightly with salt and generously with crushed red pepper, and boil 2 minutes. Pour in the vodka, lower the heat so the sauce is at a lively simmer, and simmer until the pasta is ready.\\

Just before the pasta is done, fish the garlic cloves out of the sauce and pour in the cream. Add the 2 tablespoons butter or oil, if using, and swirl the skillet to incorporate into the sauce. If the skillet is large enough to accommodate the sauce and pasta, fish the pasta out of the boiling water with a large wire skimmer and drop it directly into the sauce in the skillet. If not, drain the pasta, return it to the pot, and pour in the sauce. Bring the sauce and pasta to a boil, stirring to coat the pasta with sauce. Check the seasoning, adding salt and red pepper if necessary. Sprinkle the parsley over the pasta and boil until the sauce is reduced enough to cling to the pasta.\\

Remove the pot from the heat, sprinkle 3/4 cup of the cheese over the pasta, and toss to mix. Serve immediately, passing additional cheese if you like.\\

\end{recipe}