\begin{recipe}{\texorpdfstring{$\star$}{str}Coconut Red Lentil Soup}{6 servings - 306 calories}{}

\freeform From 101 Cookbooks.

\Ing{1 cup / 7 oz / 200g yellow split peas}
\Ing{1 cup 7 oz / 200g red split lentils (masoor dal)}
\Ing{7 cups / 1.6 liters water}
\Ing{1 medium carrot, cut into \nicefrac{1}{2}-inch dice}
\Ing{2 tbsp fresh peeled and minced ginger}
\Ing{2 tbsp curry powder}
\Ing{2 tbsp butter, ghee, or coconut oil}
\Ing{8 green onions (scallions), thinly sliced}
\Ing{3 tbsp golden raisins}
\Ing{\nicefrac{1}{3} cup / 80 ml tomato paste}
\Ing{1 14-ounce can coconut milk}
\Ing{2 tsp fine grain sea salt}
\Ing{one small handful cilantro, chopped\\}
\Ing{cooked brown rice or farro, for serving (optional)}

Give the split peas and lentils a good rinse - until they no longer put off murky water. Place them in an extra-large soup pot, cover with the water, and bring to a boil. Reduce heat to a simmer and add the carrot and \nicefrac{1}{4} of the ginger. Cover and simmer for about 30 minutes, or until the split peas are soft.\\

In the meantime, in a small dry skillet or saucepan over low heat, toast the curry powder until it is quite fragrant. Be careful though, you don't want to burn the curry powder, just toast it. Set aside. Place the butter in a pan over medium heat, add half of the green onions, the remaining ginger, and raisins. Saute for two minutes stirring constantly, then add the tomato paste and saute for another minute or two more.\\

Add the toasted curry powder to the tomato paste mixture, mix well, and then add this to the simmering soup along with the coconut milk and salt. Simmer, uncovered, for 20 minutes or so. The texture should thicken up. Simmer longer for a thicker consistency.\\

Serve over ~1/2 cup of warm farro or brown rice (+108 cal). Sprinkle each bowl generously with cilantro and the remaining green onions.

\end{recipe}