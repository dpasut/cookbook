\begin{recipe}{Bacon}{varies by weight}{}
    \freeform From Joshua Weissman How to Make The Best Homemade Bacon.

    \Ing{\textbf{Large batch Dry Cure:}}
    \Ing{400g kosher salt}
    \Ing{200g granulated sugar}
    \Ing{60g pink curing salt}
    \Ing{\textbf{Bacon:}}
    \Ing{$1\nicefrac{1}{2}$ kg (or larger) skin on, pork belly }
    \Ing{3 cloves garlic}
    \Ing{2 sprigs rosemary}

    \textbf{Creating the Cure}\\
    Combine salt, sugar and curing salt and whisk to combine well. Store in a sealed mason jar.\\

    \textbf{Cure the pork belly}\\
    Cure the pork belly at a 2.5 percent cure. Multiply the weight of your pork belly in grams by .025 and the number you get from that will be the amount of dry cure that you need to use on the bacon. So in other words a 1600g pork belly would require 40g of dry cure.\\

    Cover the pork belly in the cure and place in a plastic bag (add garlic and rosemary, or other aromatics of choice) for 7 to 10 days, flipping half way through.\\

    Rinse off and pat dry.\\

    \textbf{Cook the Pork Belly}\\

    Cook at $200\degree$F until the internal temperature reaches $150\degree$F. This can be done in an oven on a rack above a baking sheet, or in a smoker. This will take between 2 and 4 hours likely, depending on the size of the pork belly and consistency of the heat.\\

    \textbf{Cool and Slice}\\

    Let the pork belly cool at room temperature for about 10 minutes, until its cool enough to handle. Carefully remove the skin using a knife. Once removed, wrap the pork belly tightly in plastic wrap and refrigerate overnight.\\

    Slice to your desired thickness. To store, freeze on parchment lined baking sheets in a single layer. Once frozen, remove from baking sheets and put in a ziplock bag. Store for up to 3 months in the freezer.

\end{recipe}