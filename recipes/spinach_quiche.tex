\begin{recipe}{\texorpdfstring{$\star$}{str}Spinach Quiche}{12 servings - 358 calories}{}

\freeform From Smitten Kitchen.

\Ing{\textbf{Crust:}}
\Ing{1\nicefrac{2}{3} cup (215 g) all-purpose flour}
\Ing{\nicefrac{3}{4} tsp kosher salt}
\Ing{12 tbsp (170 g) cold unsalted butter, diced}
\Ing{3 tbsp (45 g) very cold water\\}
\Ing{\textbf{Filling:}}
\Ing{Nonstick spray oil, for coating pie dish}
\Ing{\nicefrac{3}{4} (176 g, 6 oz, or \nicefrac{3}{4} of an 8-oz brick) cream cheese, soft at room temperature}
\Ing{\nicefrac{2}{3} cup (155 ml) half-and-half or 1/3 cup each whole milk and heavy cream}
\Ing{6 large eggs}
\Ing{2 10-oz (283-g) packages frozen chopped spinach, thawed}
\Ing{1 cup (115 g or 4 oz) grated cheddar or Gruyere}
\Ing{\nicefrac{1}{2} cup (50 g) finely grated Parmesan}
\Ing{1 small bundle (2 to 3 oz or about 8 thin green onions) thinly sliced}
\Ing{1 tsp kosher salt}
\Ing{\nicefrac{1}{2} tsp freshly ground black pepper}

\textbf{Crust}\\

In a food processor, blend flour and salt together. Add butter and pulse machine until butter is reduced to a fine meal, or couscous-sized bits. While running machine, drizzle in water; stop when dough has balled, a few seconds later.\\

Wrap dough in plastic or waxed paper and set in freezer to quick-chill until firm but not rock-hard, about 15 to 20 minutes. You can also chill it in the fridge for 2 hours or up to 1 week until needed.\\

Lightly coat a deep pie dish with oil.\\

To roll out crust: Flour your counter well. Remove crust from freezer or fridge, unwrap and flour the top of it. Even if it’s very hard, begin rolling it very gently, in light motions, so it doesn't crack too much as you stretch it out. Keep flouring top and counter underneath dough as it is prone to sticking. Work as quickly as possible because this dough softens even more than regular pie dough as it warms.\\

Transfer dough to prepared pie dish. Trim overhang. To press in crust: Press dough in an even layer across bottom of dish. Freeze shaped dough until solid, about 20 minutes. Save your scraps. You can use them to patch any holes or cracks formed when baking.\\

\newpage

Meanwhile, heat oven to 425 degrees F. Coat a large sheet of foil lightly with spray oil. Once crust is solid, prick it all over with a fork and press foil, oiled side down, tightly against dough. Fill foiled crust to the top with pie weights, dried beans or rice (that you don’t plan to eat at any time) or even pennies. Bake for 20 minutes then gently, carefully remove foil and weights and bake for 5 more minutes, unfilled.\\

\textbf{Filling}\\

Use an electric mixer or your best whisking skills to beat cream cheese in the bottom of a large bowl until smooth and fluffy. Gradually drizzle in half-and-half, whisking the whole time so that the mixture incorporates smoothly. Whisk in eggs, two at a time, until combined. Squeeze out spinach in handfuls, removing as much extra moisture as possible. Stir in spinach, cheddar, Parmesan, scallions, salt and pepper.\\

\textbf{Quiche}\\

When crust has finished par-baking, leave oven on. Inspect crust for cracks or holes and use reserved dough to patch them if necessary. Pour in filling just to the top of the crust. You may have more filling than you can fit in the crust; you can bake this off in a separate oiled dish for an excellent breakfast on toast tomorrow.\\

Bake quiche until crust is golden brown and filling is set, about 25 minutes. Cool at least 10 minutes before serving. Quiche keeps in fridge for 4 to 5 days.

\end{recipe}