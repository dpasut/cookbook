\begin{recipe}{Basic Almond Cheese}{12 servings - 109 calories}{}

\freeform From Land \& Flavors.

\Ing{160 g or 5.5 oz unroasted, blanched almonds, soaked for 6 hours or more (1\nicefrac{1}{4} cups of whole blanched almonds or 1\nicefrac{3}{4} cups of almond flour/meal.)}
\Ing{2\nicefrac{1}{2} tablespoons (36 ml) lemon juice}
\Ing{3 tbsp (45 ml) extra virgin olive oil}
\Ing{\nicefrac{1}{2} clove of garlic}
\Ing{1\nicefrac{1}{4} teaspoons sea salt}
\Ing{\nicefrac{2}{3} cup (160 ml) water}

Soak the almonds in water for 6 hours or overnight.\\

Drain and put the almonds into a blender with the lemon juice, olive oil, garlic, salt, and water.
Blend on high until the almonds become as smooth as possible. Depending on your blender, this may take a few minutes. Pause and scrape down the sides of the blender occasionally. If it is too thick and is not blending, add more water 1 Tablespoon at a time until the mixture blends properly.\\

When smooth and creamy, remove the mixture from the blender into a small sieve that has been lined with fine cheesecloth. Place this sieve over a bowl to catch the draining water and refrigerate overnight. This not only removes the excess water, but also improves the flavor by allowing the flavors to marry.\\

After the cheese has drained well overnight, carefully invert it onto a lightly oiled baking sheet, removing all of the cheesecloth. You have two options for baking: Bake at 325°F (165°C) for 25-30 minutes for a just set, more spreadable cheese. Alternatively, bake it at 350°F (180°C) for 30-40 minutes for a more set, more crumbly, yet still creamy cheese. You can even keep baking it longer at this temperature for a browned look. It may crack slightly but the flavor will still be great and the cheese will even be sliceable. I tend to prefer the hotter, longer baking method.\\

After it cools down, put in an airtight container and store it in the refrigerator. It will firm up a little after chilling.

\end{recipe}