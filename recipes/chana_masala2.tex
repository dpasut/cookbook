\begin{recipe}{\texorpdfstring{$\star$}{str}Chana Masala}{4 servings - 238 calories}{}

\freeform From The Spruce Eats by Petrina Verma Sarkar.

\Ing{3 large onions (sliced thin, divided)}
\Ing{2 large tomatoes (chopped)}
\Ing{3 tbsp ginger garlic paste paste (see \ref{components:gingergarlicpaste})}
\Ing{2 tbsp vegetable oil}
\Ing{2 bay leaves}
\Ing{5 to 6 cloves}
\Ing{3 to 4 green cardamoms}
\Ing{5 to 6 peppercorns}
\Ing{1 tsp cumin powder}
\Ing{2 tsp coriander powder}
\Ing{\nicefrac{1}{2} tsp red chili powder}
\Ing{\nicefrac{1}{4} turmeric powder}
\Ing{2 tsp garam masala}
\Ing{2 cans of chickpeas (or dired equivalent)}
\Ing{salt (to taste)}
\Ing{Water (enough to make a gravy)}
\Ing{1-inch piece of ginger (julienned)}
\Ing{2 tbsp fresh coriander leaves (chopped fine)}

Grind 2 of the sliced onions, the tomatoes, and the ginger and garlic paste together into a smooth paste in a food processor. Heat the vegetable oil in a deep, thick-bottomed pan on medium heat. Add the bay leaves, cloves, cardamom, and peppercorns and sauté until slightly darker and mildly fragrant. Add the remaining sliced onion and fry until light golden in color.\\

Add the onion-tomato paste you made earlier and fry till the oil begins to separate from the paste. Add the dry, powdered spices—cumin, coriander, red chili, turmeric, and garam masala powders. Sauté, stirring frequently, for 5 more minutes.\\

Drain the water in the can from the chickpeas and rinse them well under running water. Now add the chickpeas to the masala you fried up earlier. Stir to mix everything well. Add salt to taste and enough hot water to make the gravy—about 1 1/2 cups. Simmer and cook covered for 10 minutes.\\

Use a flat spoon or potato masher to mash some of the chickpeas coarsely. Stir to mix everything well. Garnish with juliennes of ginger and finely chopped fresh coriander leaves. A squeeze of lemon and a handful of very finely chopped onion tastes great as a garnish too. Serve hot and enjoy!

\end{recipe}