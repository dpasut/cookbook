\begin{recipe}{\texorpdfstring{$\star$}{str}Easy Chana Masala}{4 servings - 246 calories}{}

\freeform From Smitten Kitchen. For a more authentic version see \pageref{main:chana_masala2}.

\Ing{1 tbsp vegetable oil}
\Ing{2 medium onions, minced}
\Ing{1 clove garlic, minced}
\Ing{2 tsp grated fresh ginger}
\Ing{1 fresh, hot green chili pepper, minced}
\Ing{1 tbsp ground coriander}
\Ing{2 tsp ground cumin}
\Ing{\nicefrac{1}{2} teaspoon ground cayenne pepper}
\Ing{1 tsp ground turmeric}
\Ing{2 tsp cumin seeds, toasted and ground}
\Ing{1 tbsp amchoor powder, or increase the lemon juice}
\Ing{2 tsp paprika}
\Ing{1 tsp garam masala}
\Ing{2 cups tomatoes, chopped small or 1 15-ounce can of whole tomatoes with their juices, chopped small}
\Ing{\nicefrac{2}{3} cup water}
\Ing{4 cups cooked chickpeas or 2 (15-ounce) cans chickpeas, drained and rinsed}
\Ing{\nicefrac{1}{2} teaspoon salt}
\Ing{\nicefrac{1}{2} lemon (juiced), use a whole lemon if not using amchoor powder}

Heat oil in a large skillet. Add onion, garlic, ginger and pepper and sauté over medium heat until browned, about 5 minutes.\\

Turn heat down to medium-low and add the coriander, cumin, cayenne, turmeric, cumin seeds, amchoor (if using it), paprika and garam masala. Cook onion mixture with spices for a minute or two, then add the tomatoes and any accumulated juices, scraping up any bits that have stuck to the pan.\\

Add the water and chickpeas. Simmer uncovered for 10 minutes, then stir in salt and lemon juice.

\end{recipe}