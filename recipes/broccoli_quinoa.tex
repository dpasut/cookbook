\begin{recipe}{Double Broccoli Quinoa}{4 servings - 499 calories}{}

\freeform From 101 Cookbooks.

\Ing{3 cups cooked quinoa}
\Ing{5 cups raw broccoli, cut into small florets and stems\\}
\Ing{3 medium garlic cloves}
\Ing{\nicefrac{2}{3} cup sliced or slivered almonds, toasted }
\Ing{\nicefrac{1}{3} cup freshly grated Parmesan}
\Ing{2 big pinches salt}
\Ing{2 tbsp fresh lemon juice}
\Ing{\nicefrac{1}{4} cup olive oil}
\Ing{\nicefrac{1}{4} cup heavy cream\\}
\Ing{Optional extra toppings: slivered basil, fire oil, sliced avocado,
crumbled feta or goat cheese}

Heat the quinoa and set aside.\\

Now barely cook the broccoli by pouring 3/4 cup water into a large pot and bringing it to a simmer. Add a big pinch of salt and stir in the broccoli. Cover and cook for a minute, just long enough to take the raw edge off. Transfer the broccoli to a strainer and run under cold water until it stops cooking. Set aside.\\

To make the broccoli pesto puree two cups of the cooked broccoli, the garlic, 1/2 cup of the almonds, Parmesan, salt, and lemon juice in a food processor. Drizzle in the olive oil and cream and pulse until smooth.\\

Just before serving, toss the quinoa and remaining broccoli florets with about 1/2 of the broccoli pesto. Taste and adjust if needed, you might want to add more of the pest a bit at a time, or you might want a bit more salt or an added squeeze of lemon juice. Turn out onto a serving platter and top with the remaining almonds, a drizzle of the chile oil, and some sliced avocado or any of the other optional toppings.\\

\textbf{Notes}

Quinoa can be substituted for barley.\\

To cook quinoa: rinse one cup of quinoa in a fine-meshed strainer. In a medium saucepan heat the quinoa, two cups of water (or broth if you like), and a few big pinches of salt until boiling. Reduce heat and simmer until water is absorbed and quinoa fluffs up, about 15 minutes. Quinoa is done when you can see the curlique in each grain, and it is tender with a bit of pop to each bite. Drain any extra water and set aside.\newpage

To make the red chile oil: You'll need 1/2 cup extra-virgin olive oil and 1 1/2 teaspoons crushed red pepper flakes. If you can, make the chile oil a day or so ahead of time by heating the olive oil in a small saucepan for a couple minutes - until it is about as hot as you would need it to saute some onions, but not so hot that it smokes or smells acrid or burned. Turn off the heat and stir in the crushed red pepper flakes. Set aside and let cool, then store in refrigerator. Bring to room temp again before using.

\end{recipe}