\begin{recipe}{Paneer}{4 portions - 230 calories}{}

    \freeform From Indian Healthy Recipes by Swasthi.

    \Ing{1\nicefrac{1}{2} liters full fat milk}
    \Ing{1\nicefrac{1}{2}-2 tbsp lemon juice}

    Bring milk to boil in a heavy bottom pot. When the milk comes to a boil, pour lemon juice or curd or vinegar. Then gently stir the milk. Immediately you can see the entire milk begins to curdle. Turn off the flame. At this stage if your milk doesn't curdle, pour some more lemon juice immediately to curdle the milk. If you continue to cook at this stage, paneer can be hard, so to prevent that you need to use a bowl full of ice cubes or ice cold water to stop it from getting cooked further.\\

    Allow it to settle for 1 min and pour it in a thin cheese cloth lined over a colander. Rinse it under running water to remove the smell of the lemon juice.\\

    Make a knot to the cheese cloth. Squeeze any excess water and hang it for 30 mins to remove excess whey. Place muslin cloth on a plate with holes, place a heavy object on it for the paneer to set. After 1 hour, Remove the cloth and cut it to cubes. Refrigerate the paneer and use up with 2 to 3 weeks. Or freeze up to 3 months.\\

    \textbf{Notes:}\\

    Milk may not curdle if the lemon juice is not suffcient, immediately you must add little more lemon juice/vinegar in that case. Adding too much of lemon juice will make your paneer harder, use just as needed.\\

    The resulting liquid, whey, is great for baking bread! Use it instead of water for a more flavourful, chewier bread (see p. \pageref{bread:noknead}, p. \pageref{bread:focaccia},p. \pageref{components:pizza_dough}).\\

    Do not overcook the paneer. It can become hard. As soon as the milk begins to curdle switch off the flame.

\end{recipe}