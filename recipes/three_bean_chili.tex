\begin{recipe}{\texorpdfstring{$\star$}{str}Three-Bean Chili}{6 servings - 326 calories}{}

\freeform From Smitten Kitchen.

\Ing{1 tbsp olive oil}
\Ing{1 medium onion, chopped small}
\Ing{1 to 2 peppers of your choice (see Notes, below), finely chopped}
\Ing{3 cloves garlic, minced}
\Ing{2 tbsp chili powder}
\Ing{2 tsp ground cumin}
\Ing{1 tsp dried oregano}
\Ing{1 \nicefrac{1}{2} tsp table salt or 2 \nicefrac{1}{2} tsp kosher or coarse salt}
\Ing{1 12-ounce bottle beer}
\Ing{1 28-ounce can crushed tomatoes, fire-roasted if you can find them}
\Ing{\nicefrac{1}{2} cup dried kidney beans}
\Ing{\nicefrac{1}{2} cup dried black beans}
\Ing{\nicefrac{1}{2} cup dried pinto beans}
\Ing{3 \nicefrac{1}{2} to 4 cups water\\}
\Ing{To serve: Lime wedges, sour cream, diced white onion, cilantro, corn or flour tortillas or tortilla chips or rice}

Heat oil in the bottom of a medium-sized heavy pot or Dutch oven (if finishing it on the stove) or in a large skillet (if finishing in a slow-cooker). Once warm, add onion and cook for 5 minutes, until translucent. Add any fresh peppers and cook for 3 more minutes. Add garlic, chili powder, cumin, oregano and salt and cook for 2 minutes, until browned and deeply fragrant. Add beer and scrape up any bits stuck to the pot. Boil until reduced by half.\\

If finishing on the stove: Add tomatoes, dried beans, any dried or rehydrated-and-pureed chilies and the smaller amount of water. Bring mixture to a full boil and boil for one minute, then reduce heat to a very low, gentle simmer, place a lid on your pot, and cook for 2 1/2 to 3 hours, until the beans are tender, stirring occasionally. Add the last 1/2 cup water if mixture seems to be getting dry.\\

If finishing in a slow-cooker: Scrape onion, spice and beer mixture into a slow-cooker and add tomatoes, dried beans, any dried or rehydrated-and-pureed chilies and the smaller amount of water. Cook on HIGH for 6 to 7 hours, until beans are tender. You can add the last 1/2 cup water if needed, but probably will not find it necessary.\\

Serve as-is or with fixings of your choice.\newpage

\textbf{Notes}

Peppers: The most important decision you make about your chili is, unsurprisingly, in the chilies themselves. If you're cooking for people who don't like spicy food, I recommend just using 1 bell pepper or 1 fresh poblano, which is very mild. 2 fresh jalapenos will give you slightly more heat. 2 small dried chilies, depending on which you use, will give you a bit more of a kick, as will 1 to 2 chipotle en adobo peppers from a can. To best incorporate the flavor of dried chilies into your chili, cover them with a bit of boiling water until they’re soft, then puree them. If this sounds like too much work, you can cook them with the dried beans for decent heat flavor infusion.\\

Chili powder: If you’d like the clear flavor of your dried chilies to come through, you can skip the chili powder in part or entirely.\\

Using canned beans instead: 1 1/2 cups dried beans will yield approximately 3 to 3 3/4 cups of cooked ones. To use canned or already-cooked beans instead, you’ll want to use 2 to 3 15-ounce cans of cooked beans and then — this is important — skip the water. Simmer all of the ingredients except the drained and rinsed beans for 20 minutes, then add the beans and simmer it 10 minutes more. If the mixture looks dry, add 1/4 cup water and simmer for another few minutes.

\end{recipe}