\begin{recipe}{Chocolate Blackout Cake}{12 slices - 693 calories ea.}{}

\freeform From {\normalfont The Complete Canadian Living Cookbook} by Elizabeth Baird (p.303).

\Ing{\textbf{Cake}}
\Ing{2 cups granulated sugar}
\Ing{2 cups water}
\Ing{4 oz, 125 g unsweetened chocolate, chopped}
\Ing{\nicefrac{1}{3} cup butter}
\Ing{1 tsp vanilla}
\Ing{2 eggs, lightly beaten}
\Ing{2 cups all purpose flour}
\Ing{2 tsp baking powder}
\Ing{2 tsp baking soda}
\Ing{\nicefrac{1}{2} tsp salt\\}
\Ing{\textbf{Icing}}
\Ing{1 \nicefrac{1}{2} cups granulated sugar}
\Ing{1 \nicefrac{1}{3} cups whipping cream}
\Ing{6 oz, 175g unsweetened chocolate, chopped}
\Ing{\nicefrac{2}{3} cup butter, softened}
\Ing{1 tsp vanilla}

\textbf{Cake}

Grease two 8 inch metal cake pans.\\

In saucepan bring sugar and water to boil. Stir until sugar dissolves.\\

Place chocolate and butter in large bowl; whisk in sugar mixture until melted and smooth; stir in vanilla. Let cool slightly. Beat in eggs.\\

In separate bowl whisk flour, baking powder, baking soda and salt. Add all at once to chocolate mixture. Beat with electric mixer until smooth. Divide equally to cake pans. Bake at 350$\degree$F for about 35 minutes until tops spring back when lightly touched. Cool in pans on racks for 30 minutes. Remove from pans and cool completely on racks.\\

\textbf{Icing}

In saucepan heat sugar with cream until just boiling. Remove from heat. Whisk
in chocolate, butter and vanilla until melted and smooth. Transfer to bowl; refrigerate for 2 hours until cold. Using electric mixer, beat for about 5 minutes until thick and glossy.\\

\textbf{To Finish}

Slice each cake horizontally. Put a layer of wax paper on cake plate. Put one layer of cake on and spread top with one heaping cup of icing. Level icing. Repeat for 3 layers, keeping one layer.\\

Refrigerate the last layer for 10 minutes. Crumble the layer and sprinkle on top and press onto sides.

\end{recipe}